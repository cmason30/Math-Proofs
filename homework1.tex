\documentclass[12pt,oneside]{article}

% This package simply sets the margins to be 1 inch.
\usepackage[margin=1in]{geometry}

% These packages include nice commands from AMS-LaTeX
\usepackage{amssymb,amsmath,amsthm}

% Make the space between lines slightly more
% generous than normal single spacing, but compensate
% so that the spacing between rows of matrices still
% looks normal.  Note that 1.1=1/.9090909...
\renewcommand{\baselinestretch}{1.1}
\renewcommand{\arraystretch}{.91}

% Define an environment for exercises.
\newenvironment{exercise}[1]{\vspace{.1in}\noindent\textbf{Exercise #1 \hspace{.05em}}}{}

% define shortcut commands for commonly used symbols
\newcommand{\R}{\mathbb{R}}
\newcommand{\C}{\mathbb{C}}
\newcommand{\Z}{\mathbb{Z}}
\newcommand{\Q}{\mathbb{Q}}
\newcommand{\N}{\mathbb{N}}
\newcommand{\calP}{\mathcal{P}}

\DeclareMathOperator{\vsspan}{span}

%%%%%%%%%%%%%%%%%%%%%%%%%%%%%%%%%%%%%%%%%%

\begin{document}

% If you use Overleaf, the name of the project will be determined by
% what you enter as the document title.
\title{Math 290 Section 1 Homework}

\begin{flushright}
\textsc{Colin Mason}  \\
Math 290 Sec 3\\
September 2, 2021
\end{flushright}

\begin{center}
\textsf{Assignment 1} \\
\textsf{Exercises: 1.1 - 1.8}
\end{center}

%%%%%%%%%%%%%%%%%%%%%%%%%%%%%%%%%%%%%%%%

\begin{exercise}{1.1}
 Each of the following sets is written in set-builder notation. Write the
set by listing its elements. Also state the cardinality of each set.


\end{exercise}

\begin{proof}[(a) $S_1 = \{n \in \N : 5 < |n| 11\}$]

\item$S_1 = \{6, 7, 8, 9, 10 \}$
\item$|S_1| = 5$

\end{proof}

\begin{proof}[(b) $S_2 = \{n \in \Z : 5 < |n| 11\}$]

\item$S_2 = \{-6, 6, -7, 7, -8, 8,-9, 9,-10, 10 \}$
\item$|S_2| = 10$

\end{proof}

\begin{proof}[(c) $S_3 = \{n \in \R : x^2 +2 = 0\}$]

\item$S_3 = \emptyset$
\item$|S_3| = 0$

\end{proof}

\begin{proof}[(d) $S_4 = \{n \in \C : x^2 +2 = 0\}$]

\item$S_4 = \{0-i\sqrt{2}, 0 + i\sqrt{2}\}$
\item$|S_4| = 2$

\end{proof}

\begin{proof}[(e) $S_5 = \{n \in \Z : t^5 < 1000\}$]

\item$S_5 = \{-3,-2,-1, 0,1,2,3\}$
\item $|S_5| = 7$

\end{proof}




%%%%%%%%%%%%%%%%%%%%%%%%%%%%%%%%%%%%%%%%

\begin{exercise}{1.2}
Rewrite each of the following sets in the form
$\{x \in S : \text{some property on x}\}$,
just as we did in (1.6) above, by finding an appropriate property


\end{exercise}

\begin{proof}[(a) $A_1 = \{1, 3, 5, 7, 9 ...\} \text{ where } S = \N$]

\item$A_1 = \{x \in S: \text{x is an odd number}\}$

\end{proof}

\begin{proof}[(b) $A_1 = \{1, 8, 27, 64 ...\} \text{ where } S = \N$]

\item$A_2 = \{x \in S: \text{x is the cube of any natural number.}\}$

\end{proof}

\begin{proof}[(c) $A_3 = \{1, 3\} \text{ where } S = \{-1, 0 1\}$]

\item$A_3 = \{x\in S: \text{x is different than 1}\}$

\end{proof}

%%%%%%%%%%%%%%%%%%%%%%%%%%%%%%%%%%%%%%%%

\begin{exercise}{1.3}
Write the following sets in set-builder notation

\end{exercise}

\begin{proof}[(a) $A = \{..., -10, -5, 0, 5, 10, 15, ...\}$]

\item$A = \{x \in \Z: 5x \}$

\end{proof}

\begin{proof}[(b) $B = \{..., -7, -2, 3, 8, 13, 18,  ...\}$]

\item$B = \{x \in \Z: 5x+3\}$

\end{proof}

\begin{proof}[(c) $C = \{1, 16, 81, 256, ...\}$]

\item$C = \{x \in \N: x^4\}$

\end{proof}

\begin{proof}[(d) $D = \{..., 1/4, 1/2, 1, 2, 4, 8, 16, ...\}$]

\item$D = \{x \in \Z: 2^{x}\}$

\end{proof}


%%%%%%%%%%%%%%%%%%%%%%%%%%%%%%%%%%%%%%%%

\begin{exercise}{1.4}
Give specific examples of sets $A$, $B$, and $C$ satisfying the following
conditions (in each part, separately):

\end{exercise}

\begin{proof}[(a) $A \in B, B \in C, A \notin C$]

\item$A = \{1\}$
\item$B = \{\{1\}\}$
\item$C = \{\{\{1\}\}\}$

\end{proof}

\begin{proof}[(b) $A \in B, B \subseteq C, A \nsubseteq C$]

\item$A = \{1\}$
\item$B = \{\{1\}\}$
\item$C = \{\{1\}\}$

\end{proof}

\begin{proof}[(c) $A \subset B, B \in C, A \in C$]

\item$A = \{1, 2\}$
\item$B = \{1, 2, 3\}$
\item$C = \{\{1, 2, 3\}, \{1, 2\}\}$

\end{proof}

\begin{proof}[(d) $A \cap B \subseteq C, A \nsubseteq C, B \nsubseteq C$]

\item$C = \{1\}$
\item$A = \{1, 2\}$
\item$C = \{1, 3\}$

\end{proof}

\begin{proof}[(e) $A \cap C = \emptyset, A \subseteq B, |B \cap C| = 3$]

\item$A = \{1\}$
\item$C = \{3,4,5,6\}$
\item$B = \{1,4,5,6\}$

\end{proof}


%%%%%%%%%%%%%%%%%%%%%%%%%%%%%%%%%%%%%%%%

\begin{exercise}{1.5}
Let $A = \{1, 2\}$. Find P(A), and then find P(P(A)). What are the cardinalities of these three sets?


\end{exercise}

\begin{proof}[Answers]
    

\item$P(A) = \{\{1, 2\}, \{2\}, \{1\}, \{\}$

\item$P(P(A)) = \{\{\{1, 2\}, \{2\}, \{1\}, \{\}\}, \{\{2\}, \{1\}, \{\}\}, \{\{1, 2\}, \{1\}, \{\}\}, \\ \{\{1\}, \{\}\}, \{\{1, 2\}, \{2\}, \{\}\}, \{\{2\}, \{\}\}, \{\{1, 2\}, \{\}\}, \{\{\}\}, \\ \{\{1, 2\}, \{2\}, \{1\}\}, \{\{2\}, \{1\}\}, \{\{1, 2\}, \{1\}\}, \{\{1\}\}, \{\{1, 2\}, \{2\}\},\\ \{\{2\}\}, \{\{1, 2\}\}, \{\}\}$

\item$|A| = 2$\item$|P(A)| = 4$\item$|P(P(A))| = 16$
\end{proof}

%%%%%%%%%%%%%%%%%%%%%%%%%%%%%%%%%%%%%%%%

\begin{exercise}{1.6}
Let $a, b \in \R \text{ with } a < b$. The closed interval $[a, b]$ is the set $\{x \in \R :a \leq x \leq b\}$. Similarly, the open interval $(a, b)$ is the set $\{x \in \R : a < x < b\}$. Let $P = [3,7]$, $Q=[7,9]$, and $R = [-3, 8]$.

\end{exercise}

\begin{itemize}
    \item[(a)] $P \cap Q = \{7\}$
    \item[(b)] $P \cup Q = [3,9]$
    \item[(c)] $P - Q = \{x \in \R : 3 < x <7\}$
    \item[(d)] $Q - P = \{x \in \R : 7 < x <9\}$
    \item[(e)] $(R \cup P) - Q= \{x \in \R : 3 < x <7\}$
    \item[(f)] $(P \cup Q) \cap R = [3, 8]$
    \item[(g)] $P \cup (Q \cap R) = [3, 8]$
\end{itemize}

%%%%%%%%%%%%%%%%%%%%%%%%%%%%%%%%%%%%%%%%

\begin{exercise}{1.7}
Consider the following blank Venn diagram for the three sets $A, B, C$.

\end{exercise}

\begin{proof}[Answers at end of PDF file.]

\end{proof}

%%%%%%%%%%%%%%%%%%%%%%%%%%%%%%%%%%%%%%%%
\begin{exercise}{1.8}
Two sets $S$, $T$ are disjoint if they share no elements. In other words
$S \cap T= \emptyset$. Which of the following sets are disjoint? Give reasons.
\end{exercise}

\begin{proof}[(a) The set of odd integers and the set of even integers]

\item Disjoint. No integer can be even and odd.

\end{proof}

\begin{proof}[(b)  The natural numbers and the complex numbers.]

\item Not disjointed. Whole numbers can be complex (ex. r + 0i).

\end{proof}

\begin{proof}[(c) The prime numbers and the composite numbers.]

\item Disjoint. No number can be composite and prime.

\end{proof}

\begin{proof}[(d) The rational numbers and the irrational numbers.]

\item Disjoint. No real number can be rational and irrational.

\end{proof}
%%%%%%%%%%%%%%%%%%%%%%%%%%%%%%%%%%%%%%%%



\begin{exercise}{1.9}
Find some universal set $U$ and subsets $S,T,U \subseteq U$, such that $|S-T| = 3$, $|T-S| = 1$, $|S \cup T| = 6$, $|\overline{S}| = 2$
\end{exercise}

\begin{proof}[Answers]

\item$S=\{2,3,4,5,6\}$
\item$T= \{1,2\}$
\item$U=\{1,2,3,4,5,6,7\}$

\end{proof}


%---------------------------------
% Don't change anything below here
%---------------------------------


\end{document}