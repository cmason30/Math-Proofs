\documentclass[12pt,oneside]{article}

% This package simply sets the margins to be 1 inch.
\usepackage[margin=1in]{geometry}

% These packages include nice commands from AMS-LaTeX
\usepackage{amssymb,amsmath,amsthm}

% Make the space between lines slightly more
% generous than normal single spacing, but compensate
% so that the spacing between rows of matrices still
% looks normal.  Note that 1.1=1/.9090909...
\renewcommand{\baselinestretch}{1.1}
\renewcommand{\arraystretch}{.91}

% Define an environment for exercises.
\newenvironment{exercise}[1]{\vspace{.1in}\noindent\textbf{Exercise #1 \hspace{.05em}}}{}

% define shortcut commands for commonly used symbols
\newcommand{\R}{\mathbb{R}}
\newcommand{\C}{\mathbb{C}}
\newcommand{\Z}{\mathbb{Z}}
\newcommand{\Q}{\mathbb{Q}}
\newcommand{\N}{\mathbb{N}}
\newcommand{\calP}{\mathcal{P}}

\DeclareMathOperator{\vsspan}{span}

%%%%%%%%%%%%%%%%%%%%%%%%%%%%%%%%%%%%%%%%%%

\begin{document}

% If you use Overleaf, the name of the project will be determined by
% what you enter as the document title.
\title{Math 290 Section 1 Homework}

\begin{flushright}
\textsc{Colin Mason}  \\
Math 290 Sec 3\\
September 2, 2021
\end{flushright}

\begin{center}
\textsf{Assignment 1} \\
\textsf{Exercises: 1.1 - 1.8}
\end{center}

%%%%%%%%%%%%%%%%%%%%%%%%%%%%%%%%%%%%%%%%

\begin{exercise}{1.1}
 Each of the following sets is written in set-builder notation. Write the
set by listing its elements. Also state the cardinality of each set.


\end{exercise}

\begin{proof}[(a) $S_1 = \{n \in \N : 5 < |n| 11\}$]
\item$S_1 = \{6, 7, 8, 9, 10 \}$
\end{proof}

\begin{proof}[(b) $S_2 = \{n \in \Z : 5 < |n| 11\}$]
\item$S_2 = \{-6, 6, -7, 7, -8, 8,-9, 9,-10, 10 \}$
\end{proof}

\begin{proof}[(c) $S_3 = \{n \in \R : x^2 +2 = 0\}$]
\item$S_3 = \emptyset$
\end{proof}

\begin{proof}[(d) $S_4 = \{n \in \C : x^2 +2 = 0\}$]

\end{proof}

\begin{proof}[(e) $S_5 = \{n \in \Z : t^5 < 1000\}$]

\end{proof}




%%%%%%%%%%%%%%%%%%%%%%%%%%%%%%%%%%%%%%%%

\begin{exercise}{1.2}
Rewrite each of the following sets in the form
$\{x \in S : \text{some property on x}\}$,
just as we did in (1.6) above, by finding an appropriate property


\end{exercise}

\begin{proof}[(a) $A_1 = \{1, 3, 5, 7, 9 ...\} \text{ where } S = \N$]

\item$A_1 = \{x \in S: \text{x is an odd number}\}$

\end{proof}

\begin{proof}[(b) $A_1 = \{1, 8, 27, 64 ...\} \text{ where } S = \N$]
\item$A_2 = \{x \in S: x^3\}$
\end{proof}

\begin{proof}[(c) $A_3 = \{1, 3\} \text{ where } S = \{-1, 0 1\}$]
\item$A_3 = \{x\in S: x \leq 1\}$
\end{proof}


%%%%%%%%%%%%%%%%%%%%%%%%%%%%%%%%%%%%%%%%

\begin{exercise}{1.8}
Two sets $S$, $T$ are disjoint if they share no elements. In other words
$S \cap T= \emptyset$. Which of the following sets are disjoint? Give reasons.
\end{exercise}

\begin{proof}
The proof goes here.
\end{proof}


%---------------------------------
% Don't change anything below here
%---------------------------------


\end{document}