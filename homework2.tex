\documentclass[12pt,oneside]{article}

% This package simply sets the margins to be 1 inch.
\usepackage[margin=1in]{geometry}

% These packages include nice commands from AMS-LaTeX
\usepackage{amssymb,amsmath,amsthm}

% Make the space between lines slightly more
% generous than normal single spacing, but compensate
% so that the spacing between rows of matrices still
% looks normal.  Note that 1.1=1/.9090909...
\renewcommand{\baselinestretch}{1.1}
\renewcommand{\arraystretch}{.91}

% Define an environment for exercises.
\newenvironment{exercise}[1]{\vspace{.1in}\noindent\textbf{Exercise #1 \hspace{.05em}}}{}

% define shortcut commands for commonly used symbols
\newcommand{\R}{\mathbb{R}}
\newcommand{\C}{\mathbb{C}}
\newcommand{\Z}{\mathbb{Z}}
\newcommand{\Q}{\mathbb{Q}}
\newcommand{\N}{\mathbb{N}}
\newcommand{\calP}{\mathcal{P}}

\DeclareMathOperator{\vsspan}{span}

%%%%%%%%%%%%%%%%%%%%%%%%%%%%%%%%%%%%%%%%%%

\begin{document}

% If you use Overleaf, the name of the project will be determined by
% what you enter as the document title.
\title{Math 290 Section 2 Homework}

\begin{flushright}
\textsc{Colin Mason}  \\
Math 290 Sec 3\\
September 4, 2021
\end{flushright}

\begin{center}
\textsf{Assignment 1} \\
\textsf{Exercises: 2.1 - 2.5}
\end{center}

%%%%%%%%%%%%%%%%%%%%%%%%%%%%%%%%%%%%%%%%

\begin{exercise}{2.1}
Sketch each of the following sets in the Cartesian plane $\R^2$


\end{exercise}

\begin{proof}[Answers at end of PDF file.]

\end{proof}

%%%%%%%%%%%%%%%%%%%%%%%%%%%%%%%%%%%%%%%%

\begin{exercise}{2.2}
Let $A = \{s, t\} $ and $ B = \{0, 9, 7\}$. Write the following sets by listing
all of their elements.

\end{exercise}

\begin{itemize}
    \item[(a)] $A \times B$
    \subitem Answer: $\{(s, 0), (t, 0), (t, 9), (s, 9), (t, 7), (s, 7)\}$
    \item[(b)] $B \times A$  
    \subitem Answer: $\{(7, t), (0, s), (9, s), (9, t), (0, t), (7, s)\}$
    \item[(c)] $A^2$ 
    \subitem Answer: $\{(s, s), (t, s), (t, t), (s, t)\}$
    \item[(d)] $B^2$
    \subitem Answer: $\{(9, 0), (0, 0), (7, 0), (7, 9), (9, 9), (0, 7), (7, 7), (0, 9), (9, 7)\}$
    \item[(e)] $\emptyset \times A$
    \subitem Answer: $\emptyset$
    

\end{itemize}

%%%%%%%%%%%%%%%%%%%%%%%%%%%%%%%%%%%%%%%%

\begin{exercise}{2.3}
Answer each of the following questions with “True” or “False” and
then provide a reason for your answer.

\end{exercise}

\begin{proof}[(a) If $|A| = 3$ and $|B| = 4$, then $ |A \times B| = 7$]

\item True. If $A$ and $B$ are finite sets then $|A \times B|=|A| \cdot |B|$

\end{proof}

\begin{proof}[(b) It is always true that $A \times B = B \times A$ when $A$ and $B$ are sets.]

\item False. $A \times B \neq B \times A$

\end{proof}

\begin{proof}[(c)] 
$\emph{Assume $I$ is an indexing set, and let $S_i$ be a set for each $i \in I$. We always have}$ \\ $\emph{$\bigcap_{i\in I}S_i \subseteq \bigcup_{i\in I}S_i$}$


\item True. Given the union of sets A and intersection of sets B, the intersection will always be a subset of the union of sets.

\end{proof}

\begin{proof}[(d) There exist distinct sets $S_1, S_2, S_3, . . .$, each of which is infinite, but $\bigcap_{i=1}^\infty S_i$ has one element.]

\item True. Example: sets in ranges $(-1/n, 1/n)$ which converge on zero will have zero if intersected.

\end{proof}

\begin{proof}[(e) The set $A^4$ consists of ordered triples from $A$]

\item False. $A \times A \times A \times A$ will consist of ordered QUADS.

\end{proof}


%%%%%%%%%%%%%%%%%%%%%%%%%%%%%%%%%%%%%%%%

\begin{exercise}{2.4}
Using the notations from Example 2.10, write the following sets (possibly using intersections or unions).

\end{exercise}

\begin{proof}[(a)]
$\emph{The set of words containing all four of the letters “a,w,x,y.”}$
\begin{itemize}
    
\item[] Let $S = \{a, w, x, y\}$
\subitem Answer: $\bigcap_{\alpha \in S} W_{\alpha}$
\end{itemize}
\end{proof}

\begin{proof}[(b)]
$\emph{The set of words not containing any of the letters “s,t,u.”}$
\begin{itemize}
\item[] Let $S = \{s,t,u\}$
\subitem Answer: $\overline{\bigcup_{\alpha \in S} W_{\alpha}}$
\end{itemize}
\end{proof}

\begin{proof}[(c)]
$\emph{The set of words containing both “p,r” but not containing any of the standard vowels.} \\ \emph{(Is set empty?)}$
\begin{itemize}
    

\item[] Let $S = \{p,r\}$
\item[] Let $V = \{a,e,i,o,u\}$
\subitem Answer: $\bigcap_{\alpha \in S}W_{\alpha} \cap \overline{\bigcup_{\alpha \in V} W_{\alpha}}$   Set is most likely empty.

\end{itemize}
\end{proof}



%%%%%%%%%%%%%%%%%%%%%%%%%%%%%%%%%%%%%%%%

\begin{exercise}{2.5}
For each number $r \in \R$, consider the “parabola shifted by r” defined
as:

\begin{center}
$P_r = \{(x,y) \in \R^2: y = x^2 + r\}$
\end{center}

Describe the following sets in set-builder notation; the answer should have no reference to “r.” Also graph the sets in the Cartesian plane.

\end{exercise}

\begin{itemize}
    \item[(a)] $\bigcup_{r \in \R} P_r$
    \subitem Answer: $\{\R\}$
    \item[(b)] $\bigcup_{r > 0)} P_r$
    \subitem Answer: $\{(x,y) \in \R^2: y < x^2\}$
    \item[(c)] $\bigcup_{r \neq 0} P_r$
    \subitem Answer: $\{(x,y) \in \R^2: \text{ not in } y = x^2\}$
    \item[(d)] $\bigcap_{r \in \R} P_r$
    \subitem Answer: $\emptyset$
    \item[(e)] $\bigcap_{r > 0} P_r$
    \subitem Answer: $\emptyset$
    
\end{itemize}

\begin{proof}[Figures at end of PDF file.]

\end{proof}

%%%%%%%%%%%%%%%%%%%%%%%%%%%%%%%%%%%%%%%%



%---------------------------------
% Don't change anything below here
%---------------------------------


\end{document}